\documentclass{article}
\usepackage[utf8]{inputenc}
\usepackage[english]{babel}

\usepackage{colortbl}
\usepackage{epigraph}
\usepackage{fancyhdr}
\usepackage{graphicx}
\usepackage{hhline}

\usepackage{xcolor}
\definecolor{light-gray}{gray}{0.95}

\title{Unicode in Terminals \\
a proposal to standardizing basic Unicode features}
\author{Christian Parpart}
\date{2020-07-27 (draft, revision 0)}

\newcommand{\code}[1]{\colorbox{light-gray}{\texttt{#1}}}

\newcommand{\DECRQM}[1]{\code{CSI ? #1 \$ p}}
\newcommand{\DECSET}[1]{\code{CSI ? #1 h}}
\newcommand{\DECRST}[1]{\code{CSI ? #1 l}}

\newcommand\VtModeNum{2027}                          % Grapheme cluster mode Id
\newcommand{\GCON}{\DECSET{\VtModeNum{}}}            % DECSM for enabling grapheme cluster processing
\newcommand{\GCOFF}{\DECRST{\VtModeNum{}}}           % DECRM for disabling grapheme cluster processing
\newcommand{\GCTEST}{\DECRQM{\VtModeNum{}}} % DECRQM for requesting current grapheme cluster processing mode

\begin{document}

\maketitle

\tableofcontents

\section{History and current state}

Historically, only 7-bit characters were supported by terminals and different languages by selecting
their respective code pages.
Later on

\begin{itemize}
    \item Back in the days: 7bit ASCII text, 8bit ASCII text, many code pages for switching character set
    \item Then Unicode came, the one to rule them all. But Terminals are incompatible.
    \item Unicode UTF-8 came, could be incooperated into terminals,
\end{itemize}

TBD.

...

Is Grapheme cluster handling an issue? Only when the application makes assumptions about
the cursor placement after having sent out a sequence of Unicode codepoints that form a grapheme
cluster.

\section{Backwards Compatibility}

TBD.

basic points are:
Everything is disabled by default, so legacy apps don't break more than they
used to break already.

\section{Future Compatibility and Stability}

TBD.

\section{Mode Detection}

\GCTEST can be used to test the which mode is currently active or if this feature is not available
at all - such as with non-supporting terminals or with terminals that have this support disabled.

\section{Mode Switching}

\begin{itemize}
    \item \GCON{} for enabling grapheme clustering
    \item \GCOFF{} for disabling grapheme clustering
\end{itemize}

\section{Feature Detection}

\DECRQM{\VtModeNum} can not just be used for testing the current mode but this VT sequence will also
respond with a specific code indicating that this mode (and thus this feature) is not supported.

The \code{DA1} could be extended to also indicate support, but \code{DECRQM} is sufficient.

There is a \textbf{feature detection}  spec in the works, that could be used in the future for
detecting this feature, too.

\section{Column width of a grapheme cluster}

\begin{itemize}
    \item TODO
    \item TBD
\end{itemize}

\section{Performance Considerations}

Maybe mention "Blink's Text Stack" (or Contour's text stack) and how they deal with caching.

\section{References}

\begin{itemize}
    \item DECRQM, https://vt100.net/docs/vt510-rm/DECRQM.html
    \item DECSM, https://vt100.net/docs/vt510-rm/SM.html
    \item DECRM, https://vt100.net/docs/vt510-rm/RM.html
    \item Grapheme segmentation algorithm, URL to Unicode TR and section,
        https://unicode.org/reports/tr29/\#Grapheme\_Cluster\_Boundaries
    \item Maybe also URL to "Blink's Text Stack",
        https://chromium.googlesource.com/chromium/src/+/master/third\_party/blink/renderer/platform/fonts/README.md
        or the one from Contour:
        https://github.com/christianparpart/contour/blob/master/docs/text-stack.md
\end{itemize}

\end{document}
